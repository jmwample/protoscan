\section{Introduction}\label{sec:intro}


Internet censorship is a global problem that affects over half the world's
population. Censors rely on sophisticated network middleboxes to inspect and
block traffic, employing IP-based blocking and packet injection to prevent
access to censored content and resources. A common technique used by censors involves
inspecting network traffic
passively, and injecting (spoofed) responses to DNS, TLS, HTTP, or other protocol
requests for censored
content~\cite{lowe2007great,hoang2021great,vandersloot2018quack,xu2011internet,aryan2013internet,chai2019importance,elmenhorst2021web}.


Prior work has extensively studied this type of censorship
globally~\cite{niaki2020iclab,sundara2020censored,filasto2012ooni,razaghpanah2016exploring,kuhrer2015going,dagon2008corrupted,pearce2017global,scott2016satellite}
%
and for individual
countries~\cite{Anonymous2020:TripletCensors,USESEC21:GFWatch,aryan2013internet,ramesh2020decentralized,yadav2018light,gebhart2017internet,nabi2013anatomy}.
%TODO: add other studies that are measuring any kind of injection, not just DNS
%
%\cite{kuhrer2015going,Anonymous2020:TripletCensors,USESEC21:GFWatch,dagon2008corrupted,pearce2017global,scott2016satellite}.
These studies generally perform active measurements that test large
%of DNS resolvers for large
sets of domains in requests into censored countries, and identify forged
censorship responses from legitimate ones.
%
Unfortunately, this prior work has focused exclusively on the IPv4 Internet, in
part because scanning the IPv6 Internet for servers is
difficult~\cite{murdock2017target}, owing to its impossible-to-enumerate 128-bit
address space.
%

An IPv4-only view of censorship is problematic, because
IPv6 is becoming more widely deployed and used worldwide: over 35\% of current Internet
traffic is being served over native IPv6 connections~\cite{Google-IPv6} (and exceeds
50\% in some countries known to censor such as India~\cite{akamai-ipv6}).
However, it
is unclear if the same censorship mechanisms we know about in IPv4 traffic
also apply to the growing IPv6 Internet.
There is also reason to believe it could be
different, as prior work studying IPv6 in non-censorship contexts has shown IPv6 has
fundamentally different network performance~\cite{Dhamdhere-IMC2012}, security
policies~\cite{Czyz-NDSS2016}, and topologies~\cite{Czyz-SIGCOMM2014}
compared to the traditional IPv4 Internet.
%

In this paper, we perform the first (to our knowledge) comprehensive global measurement of
censorship on the IPv6 Internet, and compare it to IPv4 censorship.
% How we do it
To study censorship globally on both IPv4 and IPv6 networks, we focus on
detecting \emph{bidirectional censorship}, which can be easily observed from a
vantage point outside the country. In this form of censorship, a censor
passively watches network traffic for censored requests, such as a DNS query
for a blocked domain. When the censor sees such a request, they inject a response (such as a DNS
response with an incorrect answer), spoofing the source of the injected
response,
as shown in Figure~\ref{fig:probeSend}.
This type of censorship can be induced and detected from a single
vantage point outside any censoring
country~\cite{vanderslooth2018quack,collateral-dns,pearce2017global,scott2016satellite}.
While this technique does not capture other types of censorship (e.g. IP
blocking), it provides one view
of censorship that we can easily apply globally and across network types.

In particular, we randomly sample IP addresses from both IPv4 and IPv6
allocations, with the goal of finding routed-but-unused addresses in every
country. This removes the need to scan for active servers, which is difficult in
IPv6. By looking for unused addresses that don't respond to our control probes,
we can simply sample addresses that we know route into a country of interest.
%
For each IP, we send requests for several protocols (DNS, TLS, and HTTP)
containing potentially censored domains and observe any injected response. If a
country has a largely uniform response to censored probes, we can label it as
censoring for that protocol and network type (IPv4 or IPv6).

\medskip
% Results
Our results show that
some censors such as Tanzania and Turkey only support censorship of
their IPv4 Internet,
while others including China and Iran support both IPv4 and IPv6 censorship.
However, even censors that censor both IPv4 and IPv6 may have subtle differences
between the two: the censorship may apply to fewer networks, may miss certain
kinds of tunneling, apply to different
protocols, or to different domains or resources. These differences may
potentially be useful to circumvention researchers, providing information about
the censorship infrastructure and ways to get around it.


\if0
In this paper, we perform the first comprehensive global measurement of DNS
censorship on the IPv6 Internet. We leverage a recent network measurement
technique that can discover dual-stack IPv6 open resolvers from their IPv4
counterpart~\cite{hendriks2017potential}, and use these IPv4-IPv6 resolver pairs
to study DNS censorship globally. %By sending DNS queries to both the IPv4 and
%IPv6 interfaces of the \emph{same resolver}, 
We then use this data to measure the difference in censorship on IPv4 and IPv6.

%\medskip
%
% \paragraph{Why study IPv6 censorship?}

While it may seem that censors either do or don't support detecting and
censoring IPv6 DNS in an all-or-nothing fashion, we find that there is a
tremendous range of how well a censor blocks in IPv6 compared to IPv4.
%
In particular, although nearly all of the countries we study have some support
for IPv6 censorship, we find that most block less effectively in IPv6 compared
to IPv4. For instance, we observe Thailand censors on average 80\% fewer IPv6
DNS resources compared to IPv4 ones, despite a robust nation-wide censorship
system~\cite{gebhart2017internet}.

Studying censorship in IPv6 can provide opportunities for circumvention tools.
By identifying ways that censors miss or incorrectly implement blocking, we can
offer these as techniques that tools can exploit. Moreover, because of the
complex and heterogeneous censorship systems censors operate, many of these
techniques would be costly for censors to prevent, requiring investing
significant resources to close the IPv4/IPv6 gap in their networks. For this
reason, we believe IPv6 can provide unique techniques for circumvention
researchers and tool developers alike, that will be beneficial in the short term
and potentially robust in the longer term.

\medskip
% \paragraph{Findings}
We find a significant global presence of IPv6 DNS censorship --- comparable, but
not identical to well documented IPv4 censorship efforts. Censors demonstrate a
clear bias towards IPv4, censoring \texttt{A} queries in IPv4 at the highest
rates, and a propensity for censoring native record types (\texttt{A} in IPv4,
\texttt{AAAA} in IPv6). At the country level we break down differences by
resolver and domain across resource record and interface type. We find that
multiple countries --- Thailand, Myanmar, Bangladesh, Pakistan, and Iran ---
present consistent discrepancies across all resolvers or domains indicating
centrally coordinated censorship, where the policies that govern IPv4 and IPv6
censorship are managed centrally. Other countries show more varied discrepancies
in the ways that resolvers censor IPv4 and IPv6, due to decentralized models of
censorship, such as that in Russia~\cite{ramesh2020decentralized}, or due to
independent and varied corporate network firewalls.
%
We also identify behavior indicative of censorship oversight that can be
advantageous to censorship circumvention. For example Brazil and Thailand censor
IPv6 queries that rely on 6to4 bridges at lower rates, presumably due to the
encapsulation of an IPv6 DNS request in an IPv4 packet, instead of appearing as
UDP.

Taken all together, this study provides a first look at IPv6 DNS censorship and
the policy gaps that arise from the IPv6 transition. We provide the following
contributions:

\begin{itemize}
    \item
    %We perform the first comprehensive study on IPv6 DNS censorship
    We conduct the first large-scale measurement of IPv6 DNS censorship in over
    100~IPv6-connected countries. We find that while most censors support IPv6
    in some capacity, there are significant gaps in how well they censor IPv6.

    \item We provide methodological improvements on measuring DNS censorship
    that avoids relying on cumbersome IP comparisons (that are not
    robust to region-specific DNS nameservers). Our methods are easily
    reproducible, and can be used in future measurement studies.

    \item We characterize the difference in censorship of both network type
    (IPv4 and IPv6), and resource type ({\tt A} and {\tt AAAA} record), and
    identify trends in several countries.

    \item Using our findings, we suggest several new avenues of future
    exploration for censorship circumvention researchers, and censorship
    measurements.

\end{itemize}

The remainder of this paper is organized as follows. \Cref{sec:background} provides
background information in DNS censorship, and the relation of IPv6 to relevant DNS
infrastructure. We outline our compiled methodology and ethical design
considerations in \Cref{sec:methodology} before presenting our findings on the
global prevalence of IPv6 censorship in \Cref{sec:prevalence}. We then dig into
per country analysis based on Resource Record types in \Cref{sec:resources} and
IP protocol version in \Cref{sec:infrastructure}. We select several case studies
to highlight in \Cref{sec:cases} before covering related work in
\Cref{sec:related}. Finally \Cref{sec:discussion} provides discussion and
contextualization of this work before concluding.
\fi

% Censorship is bad
% Censorship measurements don't hurt
% We study censorship in the context of IPv6. Why?
% -Not all or nothing: censors can deploy v6 blocking, but be worse at it
% -Not short term: censors may have to invest significant resources to be better
% -Given above, circumvention tools can benefit
% High level takeaways: some censors bad at IPv6
% -Clear bias toward IPv4, though most censors have some IPv6 capabilities
% -
