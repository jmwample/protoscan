\section{Discussion and Conclusions} \label{sec:discussion}

In this section, we detail limitations
(\Cref{sec:discussion:limitations}), directions for future research
(\Cref{sec:discussion:future}), and the takeaways of our work.
(\Cref{sec:discussion:conclusions}).

% The design and evaluation of this experiment included many complicating factors.
% Our decision making was structured to achieve the conservative result at each
% stage. However, we provide a discussion of the limitations of our experiments in
% order to transparently contextualize our work.
%
% This context provides an interesting vantage into future directions for pursuing
% censorship circumvention tools that leverage the gap in the transition from IPv4
% to IPv6. The inclusion of resolvers relying on 6to4 bridges demonstrates that
% the challenge facing censors in thoroughly covering the transition from IPv4 to
% IPv6 is non-trivial. In this section we evaluate the limitations of our
% measurements and analyses before proposing related future research directions
% and concluding.

\subsection{Limitations}
\label{sec:discussion:limitations}
Measuring censorship in order to gain an understanding of the underlying
infrastructure and identify weaknesses for circumvention is a challenging task
due to the absence of ground truth for validation and the often probabilistic
nature of censorship and networking failures which are easily confused.

Although we take care to always err on the side of caution and consider many
confounding factors including end-point type and AS diversity, our work is
fundamentally a best-effort attempt at trying to identify the gaps that have
emerged in Bidirectional censorship deployments because of the increased
adoption of IPv6.

\para{External sources of data.}
Our study relies on multiple data sources including \textit{The Citizen
Lab}~\cite{TheCitiz6:online} for our domain lists, the \textit{Route Views
Project}~\cite{RouteVie20:online} for BGP allocation data, and Maxmind's
datasets~\cite{maxmind-connectiondb} for geolocating our chosen target
addresses. Although each of these datasets has been validated in the past and
are commonly used in research, our results and their corresponding analyses are
limited by their reliability.

\para{Network stability}
Due to the nature of bidirectional censorship, where the typical benign response
is no response, it is not possible to easily distinguish a negative result from
a probe that would receive a censorship response but was dropped by the network
before it reached the censoring link. We believe that our results are still
representative as we are looking for the presence of bidirectional censorship
capabilities in aggregate rather than relying on the correctness each individual
probe. The drop rates in most networks are low and each allocation has 10
selected addresses, each of which receives a probe for each of the 1400 domains
that we test providing a significant level of redundancy.

% \para{Statistical limitations.} \dots
% Throughout our study, we aimed to err on the side of caution in order to avoid
% presenting false-positives in our results. Therefore, we relied on rigorous
% statistical approaches in order to identify differences in censorship between
% {\tt A/AAAA} query and {IPv4/IPv6} network types. This included grouping
% related hypotheses together and performing Sidak corrections to ensure that the
% confidence level achieved across the entire group (rather than for each
% individual hypothesis) was 95\%. Although this provides our results with
% credibility, our strict methods almost certainly assure false-negatives. In
% fact, this becomes apparent in our identification of {\tt A/AAAA}-resolvers in
% China. Our statistical methods identified only six resolvers with significant
% differences (\Cref{sec:resources}), however, the manual analysis performed in
% the case study (\Cref{sec:cases}) shows that a much larger number of resolvers
% performed censorship over a small set of just 21 domains. This is a fundamental
% limitation of any test of two proportions such as the $z$-test used in this
% work.

% %   Let's bring this up if they ask about it. From what I understand the stats
% %   people are in two camps. Some think this is ok and others will kill
% %   anything where the dist **might be** non-normal
% % We filter many countries out of our statistical analysis due to the low sample
% % size of available resolvers within the country. We choose specific bounds at
% % which to perform analysis using the $z$-test given that we cannot guarantee that
% % the resolver and domain entropy evaluation draws from a normal distribution.
% % With respect to available resolvers within a country we set the lower bound at
% % 25 available resolvers that pass our control tests. We believe that this
% % provides our statistical tests with appropriate power to draw significant
% % conclusions.
\subsection{Future Work} \label{sec:discussion:future}

While we find that a global scan of bidirectional censorship provides a broad
view of network interference, several open questions and opportunities for
further investigation still exist.

\para{Other Protocols}
We present results for several protocols well known to be censored at large
scales around the world, however this study is in no way an enumeration of
censorship strategies. For example, HTTP keyword based censorship is a common
strategy known to be deployed in several nation-state networks. However, the
keyword blocklists tend to be more regionally specific and significantly larger.
At the scale of target addresses that we send in this work the number of probes
becomes difficult to manage.

Along with changes relating IP versions the protocols that carry commodity
traffic change over time, as protocols are updated and improved. To this end we
did a global measurement of both Quic and DTLS (both of which are UDP variants
of the TLS protocol) by placing the domain under test in an SNI extension of a
ClientHello packet equivalent to our TLS probe. Neither protocol showed strong
signs of censorship relating to the server name.

Our Quic probes received responses from 8233 ($0.27\%$) and 1318 ($0.33\%$)
addresses respectively for IPv4 and IPv6 respectively. All of the responding
IPv6 addresses belong to cloud hosting providers Cloudflare, Fastly, and
NextDNS. A large number of the remaining IPv4 addresses were geolocated to US,
which is consistent with both the scale of allocations in the US and the
location of the parties associated with the development of the Quic protocol.
For DTLS 1396 addresses ($0.04\%$) responded to any probe all of which were in
IPv4. Most of the responding addresses were in AS 2044 which is associated with
a company offering hosting/connectivity as a service, and AS 4193 which is
associated with the State of Washington in the US. For both Quic and DTLS the
number of addresses that responded to experimental probes, but not control
probes was so small that it could be attributed to network instability or other
statistical error. We interpret this as indicative of no current bidirectional
censorship of either protocol relating to the SNI extension.


\para{Circumvention opportunities}
The incongruity that we find in censorship deployments demonstrates that there
may be opportunities to leverage the gap to circumvent network based limitations
on free speech. Censorship efficiency and distribution is not one-to-one between
IPv4 and IPv6 allowing for potential chosen path attacks for example.
Furthermore, while not explored in this work it may be possible that protocols
designed for interim or transition period between IPv4 and IPv6, such as 6-to-4
tunnelling and teredo, would go unseen by censors.


\para{Fingerprinting}
Given the large number of networks and network-actors that we measure in this
work we intend to perform a classification of censorship behaviors at the
protocol level to identify and link common censorship infrastructure and
implementation commonalities where circumvention techniques can be shared
laterally.

For target addresses that are identified as having a censor on-path, follow-up
scans using tools such as geneva~\cite{bock2019geneva} could be done to further
explore the extent to which censorship can be fingerprinted and circumvented.
Elements of such a fingerprint would include packet level details like IPID and
IPTTL of injected packets as well as censorship trigger conditions relating to
protocol validity elements like flags, checksums, and extensions.


\para{Intentional Packet Drops}
One key censorship response that we do not capture in the work is intentional
packet drops. This is not passively differentiable from the benign response in
our scan, however it is a widely deployed censorship technique. One potential
way to bridge this gap is to extend this work to measure packet drops by
incorporating an analysis of the IPID in response packets sent by the truly
benign target addresses. For targets that send TCP RST packets with a globally
incrementing IPID shared by all destination hosts analysis can indicate when a
packet to the target address was dropped in-flight as described by Ensafi
\etal~\cite{ensafi:detecting}. Again this type of analysis increases the number
of packets required to establish confidence due to noise and network
instability, but such a measurement would provide a significant extension to the
results we present in this work.


\subsection{Conclusions} \label{sec:discussion:conclusions}

%The nature of the contemporary internet is such that censorship infrastructure
%is widely deployed and commonly found in many nation-state networks around the
%world.
Many governments continue to censor the Internet.
In order to better understand the scope of this censorship, particularly with respect
to the ongoing deployment of IPv6 is effected,
%and shine a light on as many censors as possible 
we perform a global measurement of bidirectional censorship on both the IPv4 and IPv6
Internet.

We experimentally find that many networks deploy at least one form of
bidirectional censorship capability. We spotlight several countries that censor at
a seemingly national scale, and capture measurements implicating several more.
In particular, we find that while some censors support IPv6, there are others that only
censor in IPv4. In addition, there are differences between the fraction
of networks that censors can employ blocking in the respective IP versions: some
censors block more networks in IPv4, for instance, suggesting that some users
may be able to escape some or all forms of censorship simply by using IPv6 if available.

%We explore the protocols that each of these censors actively interfere in
%showing that national censorship infrastructure differs significantly by the
%various protocols that are effected and the stateful-ness of the censorship
%infrastructure. Beyond this we show that multiple national censorship strategies
%have significant differences in the effects on IPv4 and IPv6 connections with
%some demonstrating relatively similar capacity between the two, and some nations
%censoring only one IP version.

It is important to understand the current state of censorship in the context of
a developing Internet. This work contributes to a broader understanding of
global censorship and the gaps therein.
