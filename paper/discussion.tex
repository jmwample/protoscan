\section{Discussion and Conclusions} \label{sec:discussion}

In this section, we detail limitations
(\Cref{sec:discussion:limitations}), directions for future research
(\Cref{sec:discussion:future}), and the takeaways of our work.
(\Cref{sec:discussion:conclusions}).

% The design and evaluation of this experiment included many complicating factors.
% Our decision making was structured to achieve the conservative result at each
% stage. However, we provide a discussion of the limitations of our experiments in
% order to transparently contextualize our work. 
%
% This context provides an interesting vantage into future directions for pursuing
% censorship circumvention tools that leverage the gap in the transition from IPv4
% to IPv6. The inclusion of resolvers relying on 6to4 bridges demonstrates that
% the challenge facing censors in thoroughly covering the transition from IPv4 to
% IPv6 is non-trivial. In this section we evaluate the limitations of our
% measurements and analyses before proposing related future research directions
% and concluding.

\subsection{Limitations}
\label{sec:discussion:limitations}
Measuring censorship in order to gain an understanding of the underlying
infrastructure and identify weaknesses for circumvention is a challenging task
due to the absence of ground truth for validation and the often probabilistic
nature of censorship and networking failures which are easily confused.
%
Although we take care to always err on the side of caution and consider many
confounding factors including end-point type and AS diversity, our work is
fundamentally a best-effort attempt at trying to identify the gaps that have
emerged in DNS censorship deployments because of the increased adoption of
IPv6. The limitations of our study arise from three sources.

\para{External sources of data.}
Our study relies on multiple data sources including Satellite and the Citizen
Lab for our domain lists, McAfee's domain labeling services for categorizing
our data, and Maxmind's datasets for geolocating and classifying connection
types of our resolvers. Although each of these datasets has been validated in
the past and are commonly used in research, our results and their corresponding
analyses are limited by their reliability.

\para{Statistical limitations.} \dots
% Throughout our study, we aimed to err on the side of caution in order to avoid
% presenting false-positives in our results. Therefore, we relied on rigorous
% statistical approaches in order to identify differences in censorship between
% {\tt A/AAAA} query and {IPv4/IPv6} network types. This included grouping
% related hypotheses together and performing Sidak corrections to ensure that the
% confidence level achieved across the entire group (rather than for each
% individual hypothesis) was 95\%. Although this provides our results with
% credibility, our strict methods almost certainly assure false-negatives. In
% fact, this becomes apparent in our identification of {\tt A/AAAA}-resolvers in
% China. Our statistical methods identified only six resolvers with significant
% differences (\Cref{sec:resources}), however, the manual analysis performed in
% the case study (\Cref{sec:cases}) shows that a much larger number of resolvers
% performed censorship over a small set of just 21 domains. This is a fundamental
% limitation of any test of two proportions such as the $z$-test used in this
% work. 

% %   Let's bring this up if they ask about it. From what I understand the stats
% %   people are in two camps. Some think this is ok and others will kill
% %   anything where the dist **might be** non-normal
% % We filter many countries out of our statistical analysis due to the low sample
% % size of available resolvers within the country. We choose specific bounds at
% % which to perform analysis using the $z$-test given that we cannot guarantee that
% % the resolver and domain entropy evaluation draws from a normal distribution.
% % With respect to available resolvers within a country we set the lower bound at
% % 25 available resolvers that pass our control tests. We believe that this
% % provides our statistical tests with appropriate power to draw significant
% % conclusions.
\subsection{Future Work} \label{sec:discussion:future}

\begin{itemize}
	\item Keyword scanning in HTTP
	\item checksum validity scanning
	\item ECH / ESNI in TLS ClientHello
\end{itemize}


\subsection{Conclusions} \label{sec:discussion:conclusions}
