\section{Discussion and Conclusions} \label{sec:discussion}

In this section, we detail limitations
(\Cref{sec:discussion:limitations}), directions for future research
(\Cref{sec:discussion:future}), and the takeaways of our work.
(\Cref{sec:discussion:conclusions}).

% The design and evaluation of this experiment included many complicating factors.
% Our decision making was structured to achieve the conservative result at each
% stage. However, we provide a discussion of the limitations of our experiments in
% order to transparently contextualize our work.
%
% This context provides an interesting vantage into future directions for pursuing
% censorship circumvention tools that leverage the gap in the transition from IPv4
% to IPv6. The inclusion of resolvers relying on 6to4 bridges demonstrates that
% the challenge facing censors in thoroughly covering the transition from IPv4 to
% IPv6 is non-trivial. In this section we evaluate the limitations of our
% measurements and analyses before proposing related future research directions
% and concluding.

\subsection{Limitations}
\label{sec:discussion:limitations}
Measuring censorship in order to gain an understanding of the underlying
infrastructure and identify weaknesses for circumvention is a challenging task
due to the absence of ground truth for validation and the often probabilistic
nature of censorship and networking failures which are easily confused.

Although we take care to always err on the side of caution and consider many
confounding factors including end-point type and AS diversity, our work is
fundamentally a best-effort attempt at trying to identify the gaps that have
emerged in Bidirectional censorship deployments because of the increased
adoption of IPv6.

\para{External sources of data.}
Our study relies on multiple data sources including \textit{The Citizen
Lab}~\cite{TheCitiz6:online} for our domain lists, the \textit{Route Views
Project}~\cite{RouteVie20:online} for BGP allocation data, and Maxmind's
datasets~\cite{maxmind-connectiondb} for geolocating our chosen target
addresses. Although each of these datasets has been validated in the past and
are commonly used in research, our results and their corresponding analyses are
limited by their reliability.

\para{Network stability}
Due to the nature of bidirectional censorship, where the typical benign response
is no response, it is not possible to easily distinguish a negative result from
a probe that would receive a censorship response but was dropped by the network
before it reached the censoring link. We believe that our results are still
representative as we are looking for the presence of bidirectional censorship
capabilities in aggregate rather than relying on the correctness each individual
probe. The drop rates in most networks are low and each allocation has 10
selected addresses, each of which receives a probe for each of the 1400 domains
that we test providing a significant level of redundancy.

% \para{Statistical limitations.} \dots
% Throughout our study, we aimed to err on the side of caution in order to avoid
% presenting false-positives in our results. Therefore, we relied on rigorous
% statistical approaches in order to identify differences in censorship between
% {\tt A/AAAA} query and {IPv4/IPv6} network types. This included grouping
% related hypotheses together and performing Sidak corrections to ensure that the
% confidence level achieved across the entire group (rather than for each
% individual hypothesis) was 95\%. Although this provides our results with
% credibility, our strict methods almost certainly assure false-negatives. In
% fact, this becomes apparent in our identification of {\tt A/AAAA}-resolvers in
% China. Our statistical methods identified only six resolvers with significant
% differences (\Cref{sec:resources}), however, the manual analysis performed in
% the case study (\Cref{sec:cases}) shows that a much larger number of resolvers
% performed censorship over a small set of just 21 domains. This is a fundamental
% limitation of any test of two proportions such as the $z$-test used in this
% work.

% %   Let's bring this up if they ask about it. From what I understand the stats
% %   people are in two camps. Some think this is ok and others will kill
% %   anything where the dist **might be** non-normal
% % We filter many countries out of our statistical analysis due to the low sample
% % size of available resolvers within the country. We choose specific bounds at
% % which to perform analysis using the $z$-test given that we cannot guarantee that
% % the resolver and domain entropy evaluation draws from a normal distribution.
% % With respect to available resolvers within a country we set the lower bound at
% % 25 available resolvers that pass our control tests. We believe that this
% % provides our statistical tests with appropriate power to draw significant
% % conclusions.
\subsection{Future Work} \label{sec:discussion:future}

While we find that a global scan of bidirectional censorship provides a broad
view of network interference, several open questions and opportunities for
further investigation still exist.

\para{Censorship Strategies}
This study is in no way an enumeration of censorship strategies, for example
HTTP keyword based censorship is a common strategy known to be deployed in
several nation-state networks. However, the keyword blocklists tend to be more
regionally specific and significantly larger. At the scale of target addresses
that we send in this work the number of probes becomes difficult to manage.

\para{Fingerprinting}
Given the large number of networks and network-actors that we identify in this
work we intend to perform a classification of censorship behaviors at the
protocol level to identify and link common censorship infrastructure and
identify commonalities where circumvention techniques can be shared.

For target addresses that are identified as having a censor on-path, follow-up
scans using tools such as geneva~\cite{bock2019geneva} could be done to further
explore the extent to which censorship can be fingerprinted and circumvented.
Elements of such a fingerprint would include packet level details like IPID and
IPTTL of injected packets as well as censorship trigger conditions relating to
protocol validity elements like flags, checksum validity, and extensions.


\para{Intentional Packet Drops}
One key censorship response that we do not capture in the work is intentional
packet drops. This is not passively differentiable from the benign response in
our scan, however it is a widely deployed censorship technique. One potential
way to bridge this gap is to extend this work to measure packet drops by
incorporating an analysis of the IPID in response packets sent by the truly
benign target addresses. For targets that send TCP RST packets with a globally
incrementing IPID shared by all destination hosts analysis can indicate when a
packet to the target address was dropped in-flight as described by Ensafi
\etal~\cite{ensafi:detecting}. Again this type of analysis increases the number
of packets required to establish confidence due to noise and network
instability, but such a measurement would provide a significant extension to the
results we present in this work.


\subsection{Conclusions} \label{sec:discussion:conclusions}

The nature of the contemporary internet is such that censorship infrastructure
is widely deployed and commonly found in many nation-state networks around the
world. In order to better understand the breadth of censorship, explore the
extent to which the ongoing deployment of IPv6 is effected, and shine a light on
as many censors as possible we develop a custom tool for performing rapid global
measurements of censorship.

We experimentally find that \red{XX} networks deploy at least one form of
bidirectional censorship capability, aggregating into \red{N} countries that
censor at a seemingly national scale. We explore the protocols that each of
these censors actively interfere in showing that national censorship
infrastructure differs significantly by the various protocols that are effected,
stateful-ness of the censor, and the impacts of that censorship on subsequent
connections. Beyond this we show that multiple national censorship strategies
have significant differences in the effects on IPv4 and IPv6 connections with
some demonstrating relatively similar capacity between the two, and some nations
censoring only one IP version.

It is important to understand the current state of censorship in the context of
a developing internet. This work contributes to a broader understanding of
global censorship and the gaps there-in.
