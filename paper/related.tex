\section{Related Work}\label{sec:related}

There is a significant body of previous work investigating censorship
strategies using characteristic protocol responses to probes sent from outside
of the country. In this segment we break down the contributions of those works
and their relation to the measurement that we perform herein.

\subsection{Regional or global measurements of Bidirectional censorship.}

Censorship strategies are not universal and each censor is unique to some
degree. Targeted measurement studies contribute to a better understanding of
block-list infrastructure ~\cite{ramesh2020decentralized, USESEC21:GFWatch} and
explain blocking phenomena~\cite{global2002great, Anonymous2020:TripletCensors}.
Global studies provide high level view on the use of DNS censorship
internationally providing context and and understanding of prevalence to
specific censorship techniques~\cite{vandersloot2018quack, scott2016satellite,
pearce2017global, sundara2020censored, niaki2020iclab}. Our work performs a
global measurement of DNS, HTTP, and TLS censorship through the lens of the
differences in IPv4 and IPv6 censorship deployments.

% \para{DNS}

% \para{TLS}

% \para{HTTP}

\subsection{IPv6 censorship measurement.}
Previous censorship studies primarily focus on measurements in the context of
IPv4. However, there have been several efforts to explicitly incorporate IPv6.
In March 2020 Hoang \etal began collection of DNS records injected by the Great
Firewall in order to classify the addresses provided, block-pages injected, and
the set of hostnames that receive injections~\cite{USESEC21:GFWatch}. Their
analysis investigates the commonality of addresses injected by the GFW, finding
that all injected \texttt{AAAA} responses are drawn from the reserved teredo
subnet \texttt{2001::/32}. However, because this study does not directly
compare the injection rates of A vs AAAA or differences in injection to DNS
queries sent over IPv4 versus IPv6, our efforts complement their findings and
provide a more detailed understanding of IPv6 censorship in China.
%
A 2021 investigation of HTTP keyword block-lists associated with the Great
Firewall found that results are largely the same between IPv4 and
IPv6~\cite{weinberg2021chinese}. However, the authors note that over IPv6
connections, the the Firewall failed to apply it's signature temporary 90 second
``penalty box'' blocking subsequent connections between the two hosts described
by numerous previous studies~\cite{xu2011internet,clayton2006ignoring}. This
supports our finding that for now the GFW infrastructure supporting IPv4 and
IPv6 are implemented and/or deployed independently.
