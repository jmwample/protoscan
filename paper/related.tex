\section{Related Work}\label{sec:related}

%There is a significant body of previous work investigating censorship
%strategies using characteristic protocol responses to probes sent from outside
%of the country. In this segment we break down the contributions of those works
%and their relation to the measurement that we perform herein.

%\subsection{Regional or global measurements of Bidirectional censorship.}

%Censorship strategies are not universal and each censor is unique to some
%degree.
Prior targeted censorship measurement studies contribute to a better understanding of
block-list infrastructure~\cite{ramesh2020decentralized, USESEC21:GFWatch} and
have helped to explain blocking phenomena~\cite{global2002great, Anonymous2020:TripletCensors}.
Meanwhile, global studies have also yielded higher-level views on the use of DNS censorship
around the world
%providing context and and understanding of prevalence to
%specific censorship techniques~\
\cite{vandersloot2018quack, scott2016satellite,
pearce2017global, sundara2020censored, niaki2020iclab}. However, all of these
studies have required Internet-wide scans, that are only feasible on IPv4. Thus,
there is a gap of knowledge when it comes to IPv6 censorship. Our work performs a
global measurement of DNS, HTTP, and TLS censorship through the lens of
comparing the
differences in IPv4 and IPv6 censorship deployments around the world.

% \para{DNS}

% \para{TLS}

% \para{HTTP}

%\subsection{IPv6 censorship measurement.}
While prior global censorship measurement work has been limited to IPv4, 
%Previous censorship studies primarily focus on measurements in the context of
%IPv4. However,
there have been several efforts to incorporate IPv6 or understand how censors
deal with IPv6-specific features.
In March 2020 Hoang \etal collected DNS records injected by the Great
Firewall in order to classify the addresses provided, block-pages injected, and
the set of hostnames that receive injections~\cite{USESEC21:GFWatch}. Their
analysis investigates the commonality of addresses injected by the GFW, finding
that all injected \texttt{AAAA} responses are drawn from the reserved teredo
subnet \texttt{2001::/32}. However, because this study does not directly
compare the injection rates of A vs AAAA or differences in injection to DNS
queries sent over IPv4 versus IPv6, our efforts complement their findings and
provide a more detailed understanding of IPv6 censorship in China.
%
A 2021 investigation of HTTP keyword block-lists associated with the Great
Firewall found that results are largely the same between IPv4 and
IPv6~\cite{weinberg2021chinese} using a single vantage point in China that had
both IPv4 and IPv6 connectivity. Their results corroborate ours---that China does
censor over IPv6. However, the authors note that over IPv6
connections, the the Firewall failed to apply its signature temporary 90~second
``penalty box'' blocking subsequent connections between the two hosts described
by numerous previous studies~\cite{xu2011internet,clayton2006ignoring}. This
supports our finding that at least some parts of the GFW's infrastructure supporting IPv4 and
IPv6 are implemented and/or deployed independently.


