
Internet censorship effects a significant portion of the internet today, as such
it is extremely important to find ways to measure censorship techniques that impart
minimal risk to censored netizens. One of the lowest risk strategies for measuring
censorship relies on censors displaying censorship behavior bidirectionally -
injecting uniformly for inress and egress traffic. However, no comprehensive
study has been done to determine which networks present bidirectional censorship
in IPv4 and IPv6 across protocols known to be censored. This leaves a gap in the
research understanding of where low risk measurement strategies can be
consistently applied.

In this paper we perform a comprehensive global study of bidirectional
censorship across an array of commonly censored protocols. Using a custom tool
to rapidly inject packets and a novel strategy in which we target non-responsive
hosts with probes we trigger bidirectional censorship behavior from
\red{XXX} unique ASNs in \red{N} countries across 4 protocols.

Further we observe that while nearly all censors support blocking IPv6, their
policies are inconsistent with and frequently less effective than their IPv4
censorship infrastructure. Our results suggest that supporting IPv6 censorship
is not all-or-nothing: many censors support it, but poorly.  As a result, these
censors may have to expend additional resources to bring IPv6 censorship up to
parity with IPv4. In the meantime, this affords censorship circumvention
researchers a new opportunity to exploit these differences to evade detection
and blocking.
