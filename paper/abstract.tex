
Internet censorship continues to impact billions of people worldwide,
and measurement of it remains an important focus of research.
However, most Internet censorship measurements have focused solely on the IPv4
Internet infrastructure. Yet, more clients and servers are available over IPv6:
According to Google, over a third of their users now have native IPv6 access.
% https://www.google.com/intl/en/ipv6/statistics.html

Given the slow-but-steady rate of IPv6 adoption, it is important to understand
its impact on censorship. In this paper, we measure and analyze how censorship
differs over IPv6 compared to the well-studied IPv4 censorship systems in use
today.

We perform a comprehensive global study of censorship across an array of
commonly censored protocols, including HTTP, DNS, and TLS, on both IPv4
and IPv6, and compare the results.
% TODO correct this list
We find that there are several differences in how countries censor IPv6 traffic,
both in terms of IPv6 resources, and in where and what blocklists or technology
are deployed on IPv6 networks. Many of these differences are
not all-or-nothing: we find that most censors have some capacity to block
in IPv6, but are less comprehensive or less reliable compared to their IPv4
censorship systems.

Our results suggest that IPv6 offers new areas for censorship circumvention
researchers to explore, providing potentially new ways to evade censors. As more
users gain access to IPv6 addresses and networks, there will be a need for tools
that take advantage of IPv6 techniques and infrastructure to bypass censorship.



\if0
 internet today, as such
it is extremely important to find ways to measure censorship techniques that impart
minimal risk to censored netizens. One of the lowest risk strategies for measuring
censorship relies on censors displaying censorship behavior bidirectionally -
injecting uniformly for inress and egress traffic. However, no comprehensive
study has been done to determine which networks present bidirectional censorship
in IPv4 and IPv6 across protocols known to be censored. This leaves a gap in the
research understanding of where low risk measurement strategies can be
consistently applied.

In this paper we perform a comprehensive global study of bidirectional
censorship across an array of commonly censored protocols. Using a custom tool
to rapidly inject packets and a novel strategy in which we target non-responsive
hosts with probes we trigger bidirectional censorship behavior from
\red{XXX} unique ASNs in \red{N} countries across 4 protocols.

Further we observe that while nearly all censors support blocking IPv6, their
policies are inconsistent with and frequently less effective than their IPv4
censorship infrastructure. Our results suggest that supporting IPv6 censorship
is not all-or-nothing: many censors support it, but poorly.  As a result, these
censors may have to expend additional resources to bring IPv6 censorship up to
parity with IPv4. In the meantime, this affords censorship circumvention
researchers a new opportunity to exploit these differences to evade detection
and blocking.
\fi
