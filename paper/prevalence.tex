\section{Prevalence of Bidirectional Censorship}
\PrevalenceGeneral
\label{sec:prevalence}
\para{Overview.} In this section, we focus on \emph{providing an overview of the bidirectional censorship we observe for the protocols discussed in \Cref{sec:methodology:censorship}, broken down by countries we observe it in}.
For each country and protocol pair (for which bidirectional censorship is observed), we further break down the results by whether bidirectional censorship was observed for:
\textbf{(1)} Both, IPv4 and IPv6
\textbf{(2)} Only IPv4
\textbf{(3)} Only IPv6.
\Cref{fig:prevalencegeneral} illustrates all of the breakdowns discussed above.  

\subsection{Prevalence of bidirectional censorship by protocol}
\label{sec:prevalence:proto}
For each of the 200k countries in our dataset and each of the protocols we tested (described in \Cref{sec:methodology:censorship}), we send a query to each IP from each allocation we picked (described in \Cref{subsec:selecting-ips}) for all 1400 domains (described in \Cref{sec:methodology:domains}). \textbf{We label a country as bidirectionally censoring a protocol over an IP version if we observe $> 20\%$ of the IPs (for the specific IP version) in the country as being bidirectionally censored.}

\para{DNS}
We tested bidirectional censorship of both, DNS-A and DNS-AAAA, for all 200 countries in our dataset. We observed only two countries to be bidirectionally censoring DNS queries; China and Iran. Iran censored $>70\%$ of all IPv4 addresses for DNS-A/AAAA and $>50\%$ of all IPv6 addresses for DNS-A/AAAA. On the other hand, China censors $>65\%$ of all IPv6 addresses for DNS-A/AAAA, but does not seem to censor DNS over IPv4.  

\para{HTTP}
HTTP was the most bidirectionally censored protocol. We observed bidirectional censorship of HTTP-stateful in 12 countries (and HTTTP-stateless in only 5 of these 12 countries). All of the 12 countries censor HTTP-stateful over IPv4, however, 5 of these countries do not censor HTTP-stateful over IPv6. For the 5 countries, that censor HTTP-stateless, all 5 censor over IPv4 but 2 of them do not censor over IPv6. 

\para{TLS}
We observed bidirectional TLS censorship in 7 countries. For these 7 countries, all censor TLS-stateful, however, only China censors TLS-stateless (that too only over IPv4). 5 of these countries censor TLS-stateful over both, IPv4 and IPv6, however Iran only censors TLS-stateless over IPv6 and Lebanon only does it over IPv4. 


\subsection{Case Studies}
\label{sec:prevalence:case}

\para{Tanzania}

\para{Russia}
% Specifically, we measure the global prevalence of DNS censorship that occur in
% the following four cases:
% %
% (1) a DNS {\tt A} query is sent over IPv4,
% (2) a DNS {\tt AAAA} query is sent over IPv4,
% (3) a DNS {\tt A} query is sent over IPv6, and
% (4) a DNS {\tt AAAA} query is sent over IPv6.
%
% \LEFTCIRCLE
% While much of prior work has focused on case (1), the increased adoption of
% IPv6 necessitates the analysis of cases (2-4) which are provided in our work.
% %
% In each case, we use our collected dataset (\cf \Cref{sec:methodology}) to
% summarize the base rate of censorship.

