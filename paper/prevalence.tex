\section{Prevalence of Bidirectional Censorship}
\PrevalenceGeneral
\label{sec:prevalence}
\para{Overview.} In this section, we focus on \emph{providing an overview of the bidirectional censorship we observe for the protocols discussed in \Cref{sec:methodology:censorship}, broken down by countries we observe it in}.
For each country and protocol pair (for which bidirectional censorship is observed), we further break down the results by whether bidirectional censorship was observed for:
\textbf{(1)} Both, IPv4 and IPv6
\textbf{(2)} Only IPv4
\textbf{(3)} Only IPv6.
\Cref{fig:prevalencegeneral} illustrates all of the breakdowns discussed above.  

\subsection{Prevalence of bidirectional censorship by protocol}
\label{sec:prevalence:proto}
For each of the 186 countries in our dataset and each of the protocols we tested
(described in \Cref{sec:methodology:censorship}), we send a query to each IP
from each allocation we picked (described in \Cref{subsec:selecting-ips}) for
all 1400 domains (described in \Cref{sec:methodology:domains}). \textbf{We label
a country as bidirectionally censoring a protocol over an IP version if we
observe $> 20\%$ of the IPs (for the specific IP version) in the country as
being bidirectionally censored.}

\para{DNS}
We tested bidirectional censorship of both, DNS-A and DNS-AAAA, for all 200 countries in our dataset. We observed only two countries to be bidirectionally censoring DNS queries; China and Iran. Iran censored $>70\%$ of all IPv4 addresses for DNS-A/AAAA and $>50\%$ of all IPv6 addresses for DNS-A/AAAA. On the other hand, China censors $>65\%$ of all IPv6 addresses for DNS-A/AAAA, but does not seem to censor DNS over IPv4.  

\para{HTTP}
HTTP was the most bidirectionally censored protocol. We observed bidirectional censorship of HTTP-stateful in 12 countries (and HTTTP-stateless in only 5 of these 12 countries). All of the 12 countries censor HTTP-stateful over IPv4, however, 5 of these countries do not censor HTTP-stateful over IPv6. For the 5 countries, that censor HTTP-stateless, all 5 censor over IPv4 but 2 of them do not censor over IPv6. 

\para{TLS}
We observed bidirectional TLS censorship in 7 countries. For these 7 countries, all censor TLS-stateful, however, only China censors TLS-stateless (that too only over IPv4). 5 of these countries censor TLS-stateful over both, IPv4 and IPv6, however Iran only censors TLS-stateless over IPv6 and Lebanon only does it over IPv4. 


\subsection{Case Studies}
\label{sec:prevalence:case}

\para{China}
censors bidirectionally for all 3 protocols that we tested, namely, DNS, HTTP and TLS. We queried a total of 104k IPv6 and 91k IPv4 addresses in China. For stateful HTTP and stateful TLS, we received a censored response from  $>73\%$ of IPv4 addresses and $66>\%$ of IPv6 addresses. However, for the stateless counterparts of these protocols, we received a lot less censored responses. For HTTP and TLS stateless protocols, we received censored responses from only 22\% of IPv4 addresses and 8\% from IPv6 addresses. For all the tested protocols, except stateless HTTP, a majority ($>95\%$) of the censored responses we received were \textit{RST} or \textit{RST + ACK} packets (for connection tear down). However for stateless HTTP, $>80\%$ of the censored responses we received were HTTP injections.

% Todo: future
% {\color{red} Add analysis of what the blockpages said and what allocs they belonged to (previous investigation revealed most of them were from cloud providers). Is that analysis needed?}.

We found that for DNS probes carrying either A or AAAA queries China censors at
a similar rate, $87.2\%$ of the tested IPv4 addresses and $99.3\%$ of IPv6
addresses. We note that there were no responses to our DNS control probes for
IPv6 in China.


Interestingly, one of the IPs of a vantage that we were using to make
measurements ended up on a Chinese blocklist between experiments. This resulted
in reduced rates of censorship responses, with the most significant impact on
IPv4. DNS responses for IPv4 were reduced to $15\%$ of measured addresses, while
IPv6 was only reduced to $66\%$. This suggested either that the blocking
mechanisms for IPv4 are independent or that their blocklists are de-synced.
\todo{effects on other protocols?}

% but only $15\%$ of IPv4 addresses. However, from some of our preliminary experiments, we observed that the percentage of IPs that received a censored response were roughly equal for IPv4 and IPv6. To test whether the discrepancy in our final results was from an architectural change in how China censors DNS for IPv4, we performed a couple follow-up experiments. Firstly, we wanted to test whether we observe this discrepancy due to throttling because of the high rate at which we were sending packets. To this end {\color{red} Details from Jack's experiment}. Second, we wanted to test whether the IP address we were conducting these tests from was added to a block-list. {\color{red} Details from Eric's follow-up experiment}

  
\para{Iran}
censors bidirectionally for DNS and stateless HTTP. For DNS-A and DNS-AAAA, we observe a censored response from $70\%$ of IPv4 addresses and $50\%$ of IPv6 addresses out of a total of 17k and 630 addresses respectively. Iran also censors stateful HTTP bidirectionally. We received a censored response for 50\% of IPv6 and 64\% of IPv4 addresses. All of the censorship responses from Iran were \textit{RST} packets \textbf{and} HTTP block pages. We did not find evidence of bidirectional stateless HTTP censorship in Iran {\color{red} Previous results show IR was doing bidi for stateless HTTP}.

Among stateful and stateless TLS, we only found evidence of bidirectional censorship of stateful TLS over IPv6. 33\% of all IPv6 addresses sent back a censorship response to our stateful TLS queries. However, preliminary experimentation showed that Iran censors stateless and stateful HTTP and TLS at similar rates {{\color{red} Add analysis from Hammas' follow up experiment of TLS whitelist}. 


\para{Russia}
We found negligible evidence of bidirectional censorship at the state level for Russia. Among all the protocols, stateful HTTP had the highest rate of bidirectional censorship with only 7\% of the 121k IPv4 and only 3\% of 7.7k IPv6 addresses receiving a censored response. For all other protocol and IP version pair, the censorship response rate was $<4\%$. 



\para{Tanzania}
censors bidirectionally for HTTP (stateful and stateless) but for none of the other protocols we tested. We received a censored response for $>35\%$ of all IPv4 addresses we queried in Tanzania. Almost all the censored responses we received were HTTP block-pages. Upon following up on some of the censored IP addresses by sending HTTP requests for domains that were censored for Tanzania, we found a link to the \textit{Tanzania Cyber Crimes Bill} \cite{Tanzania45:online} which was cited as the cause of the block page. This bill places restrictions on publishing adult content etc. Furthermore, we found that none of the 40 IPv6 addresses that we tested, responded with a censored response. 


% Specifically, we measure the global prevalence of DNS censorship that occur in
% the following four cases:
% %
% (1) a DNS {\tt A} query is sent over IPv4,
% (2) a DNS {\tt AAAA} query is sent over IPv4,
% (3) a DNS {\tt A} query is sent over IPv6, and
% (4) a DNS {\tt AAAA} query is sent over IPv6.
%
% \LEFTCIRCLE
% While much of prior work has focused on case (1), the increased adoption of
% IPv6 necessitates the analysis of cases (2-4) which are provided in our work.
% %
% In each case, we use our collected dataset (\cf \Cref{sec:methodology}) to
% summarize the base rate of censorship.

